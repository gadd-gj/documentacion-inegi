\title{Documento de visión y alcance }
\author{
        Gustavo U. Montes Hernandez \\
                Lic. Ingenieria de Software\\
        Universidad Veracruzana\\
        Campus Ixtaczoquitlán 
}
\date{\today}
\newpage

\documentclass[12pt]{article} 

\begin{document}
\newpage
\maketitle

\newpage \tableofcontents

\newpage \section{Visión de la solución}

Aqui veremos la vision del proyecto con sus respectivas caracteristicas
 
\subsection{Declaración de la visión}\label{ant}

Un mapa que muestre cuantos comercios pequeños hay por estado y que tipo de comercio es. Con una 
pagina que solicite un registro para poder acceder.

\subsection{Características principales}\label{on}

Un mapa en linea que cuente con las siguientes caracteristicas:
\begin{itemize}

\item Iniciar con un registro.        
\item Mostrar mapa 
\item Leer base de datos de mas de 1,000,000.00 de registros

\end{itemize}

\subsection{Suposiciones y dependencias}\label{occe}

Supuestos:
\begin{enumerate}
        \item Edad de Piedra
        \item Edad del Cobre
        \item Edad del Bronce
        \item Edad del Hierro
\end{enumerate}
\section{Alcance y limitaciones}

Aqui veremos el alcance y limitaciones de este proyecto
 
\subsection{Alcance de la versión inicial}\label{ant}

Tener un mapa que se pueda visualizar de manera online que cuente una base de datos de mas de 1,000,000 datos que separe, ceunte y asigne el numero de comercios por estado.


\subsection{Alcance de las versiones posteriores}\label{on}

Un mapa que por medio de una base de dato

\subsection{Limitaciones y exclusiones}\label{occe}

Nuestras limitaciones son aquellos comercios que no fueron registrados, la imposibilidad de realizar una busqueda de aquellos locales 

\end{document}