\title{Gestion de Riesgos }
\author{
        Citlalli Arceo \\
                Lic. Ingenieria de Software\\
        Universidad Veracruzana\\
        Campus Ixtaczoquitlán 
}
\date{\today}

\documentclass[12pt]{article}
\begin{document}
\maketitle

\addtocontents{toc}{\hfill \textbf{Página} \par}
\tableofcontents

\section{Introducción}
En cada proyecto pueden ocurrir situaciones que nos pongan en aprietos
ya que aveces se nos resulta dificil cumplir con la fecha de entrega,
nos surgen inconvenientes por eso tener un control sobre los riesgos
que pueden llegar a suceder y saber sobrellevarlos.\\ 

\subsection{Definición}
\textit{Riesgo:} Evento o circunstancia cuya probabilidad de ocurrencia es incierta, pero
que, en caso de aparecer, tiene un efecto (positivo o negativo) sobre los\\
objetivos de un proyecto.
Probabilidad de que una circunstancia adversa ocurra.\\

Los objetivos de la gestion de riesgos son identificar, controlar y eliminar las fuentes de riesgo antes de que empiecen a afectar al cumplimiento de los objetivos.\\

\section{Desarrollo}\label{desarrollo}
\subsection{Diagrama RBS}

Una estructura de desglose del riesgo garantiza un proceso completo de identificación sistemática de los riesgos con un nivel de detalle uniforme, y contribuye a la calidad y efectividad de la Identificación de Riesgos.\\

\setlength{\arrayrulewidth}{1mm}
\setlength{\tabcolsep}{5pt}
\renewcommand{\arraystretch}{1.5}

\begin{tabular}{ |p{2.5cm}||p{3cm}|p{2.5cm}|p{3cm}|  }
\hline
\multicolumn{4}{|c|}{Esquema RBS} \\
\hline
Técnico & Externo & Organización&Dirección del Proyecto\\
\hline
Requisitos   & Cliente    &Recursos&   Estimación\\
Tecnología &   Subcontratistas  & Priorización   &Planificación\\
Interaces &Condiciones climaticas & Financiación &  Control\\
Rendimiento    & Provedores &   &  Comunicación\\
Calidad &   &   &   \\
\hline
\end{tabular}\\

\textbf{Registro de Riesgo}\\
\begin{tabular}{ |p{2.5cm}||p{3cm}|p{3cm}|p{3cm}||p{3cm}|  }
\hline
\multicolumn{5}{|c|}{Ejemplo} \\
\hline
Riesgo & Prioridad & Probabilidad & Impacto & Descripcion\\
\hline
Muerte & ALTA    &MEDIA&   SUSPENSO & Un miebro del equipo muere\\
\hline
\end{tabular}\\

\textbf{Estimacion de Riesgo: Identificación}\\
Lista de control\\
En esta lista se presentan algunas de las preguntas para comprobar si\\
pueden o no afectar el desarrollo del proyecto.\\
\begin{itemize}
\item ¿El personal cuenta con las habilidades adecuadas?
\item ¿Se dispone del personal suficiente?
\item ¿Algun miembro del personal esta en otro proyecto?
\item ¿El personal esta comprometido con el proyecto?
\item ¿Se comprenden perfectamente los requisitos para el proyecto?
\end{itemize}

\begin{tabular}{ |p{4cm}||p{2.5cm}|p{6cm}|  }
 \hline
 \multicolumn{3}{|c|}{Posibles Riesgos} \\
 \hline
 Riesgo& Tipo & Descripción\\
 \hline
 Rotación del Personal& Proyecto&Personal abandona el proyecto\\
 Retraso de las Especidicaciones& Organización  & Las especificaciones de las interfaces no estaran a tiempo\\
 Perdida de vidas humanas &Externo & Una persona del proyecto murio o un familiar de este\\
 Defectos en diseño &Proyecto & El diseño no es el correcto\\
 Habilidades & Proyecto & El personal no esta del todo capacitado\\
 Enfermedad & Externo & Que el personal del proyecto enferme\\
 Tiempo & Estimacion & El tiempo requerido del proyecto este subestimado\\
 Economico & Recursos & Los integrantes del equipo no cuentan con suficiente dinero\\
 \hline
\end{tabular}

\section{Conclusions}\label{conclusions}
Aqui va la conclusion :3

\bibliographystyle{abbrv}
\bibliography{main}

\end{document}