\title{Gestion de Riesgos }
\author{
        Citlalli Itzel Arceo García \\
       	Arely Colorado Rodriguez\\
        David Freeman Cruz\\
        Gaddiel Gomez Jimenez\\
        Gustavo Uriel Montes Hernández\\
        Gustavo Romero Oltehua\\
                Lic. Ingenieria de Software\\
        Universidad Veracruzana\\
        Campus Ixtaczoquitlán 
}
\date{\today}

\documentclass[12pt]{article}
\usepackage{hyperref}

\begin{document}
\maketitle

\addtocontents{toc}{\hfill \textbf{Página} \par}
\tableofcontents

\section{Introducción}
En cada proyecto pueden ocurrir situaciones que nos pongan en aprietos
ya que aveces se nos resulta dificil cumplir con la fecha de entrega,
nos surgen inconvenientes por eso tener un control sobre los riesgos
que pueden llegar a suceder y saber sobrellevarlos.\\ 

\subsection{Definición}
\textit{Riesgo:} Evento o circunstancia cuya probabilidad de ocurrencia es incierta, pero
que, en caso de aparecer, tiene un efecto (positivo o negativo) sobre los\\
objetivos de un proyecto.
Probabilidad de que una circunstancia adversa ocurra.\\

Los objetivos de la gestion de riesgos son identificar, controlar y eliminar las fuentes de riesgo antes de que empiecen a afectar al cumplimiento de los objetivos.\\

\section{Desarrollo}\label{desarrollo}

\setlength{\arrayrulewidth}{1mm}
\setlength{\tabcolsep}{5pt}
\renewcommand{\arraystretch}{1.5}

\subsection{Identificación}
\textbf{Estimacion de Riesgo}\\
Lista de control\\
En esta lista se presentan algunas de las preguntas para comprobar si pueden o no afectar el desarrollo del proyecto.\\
\begin{itemize}
\item ¿El personal cuenta con las habilidades adecuadas?
\item ¿Se dispone del personal suficiente?
\item ¿Algun miembro del personal esta en otro proyecto?
\item ¿El personal esta comprometido con el proyecto?
\item ¿Se comprenden perfectamente los requisitos para el proyecto?\\
\end{itemize}

La siguiente tabla muestra los posibles riesgos que se han identificado en el proceso del desarrollo del proyecto.\\
\begin{tabular}{ |p{2cm}||p{3cm}|p{6cm}| }
\hline
\multicolumn{3}{|c|}{Identificación} \\
\hline
Num. & Tipo de Riesgo & Descripción\\
\hline
1  & Información &  Puede haber perdida de información al no tener un buen respaldo en un sistema de base de datos\\
2  & Equipos  & Los equipos actuales con los cuales se trabaja sufre desperfectos muy seguido, por lo que puede ocasionar retrasos en el manejo del sistema o que no cuenten con las habilidades\\
3  & Tecnología &  Falta de equipos informaticos o que dichos equipos fallen\\
4  & Externos &  Situaciones externas que pueden llegar afectar de alguna forma el proyecto\\
5  & Organización & No hay sufiencite organización en el equipo o en la información y no este listo en la fecha estipulada\\
\hline
\end{tabular}\\

\subsection{Registro de Riesgo}
La siguiente tabla nos muestra el analisis de cada uno de los riesgos identificados. Los valores de \textit{Probabilidad} e \textit{Impacto} esta compuesto por los siguientes valores:\\
\textbf{Probabilidad:} Muy Alta, Alta, Moderada, Baja, Muy Baja\\
\textbf{Impacto:} 1-Catastrófico, 2-Serio, 3-Tolerable, 4-Insignificante\\

\begin{tabular}{ |p{2.5cm}||p{3cm}|p{3cm}|p{3cm}||p{3cm}|  }
\hline
\multicolumn{5}{|c|}{Ejemplo} \\
\hline
Riesgo & Tipo de Riesgo & Probabilidad & Impacto & Descripcion\\
\hline
Muerte & EXTERNO &BAJA&  2 & Un miebro del equipo muere\\
\hline
\end{tabular}\\

\subsection{Tabla de Riesgos}
A continuación se presenta la tabla de riesgos que consideramos se pueden presentar a lo largo del proyecto\\

\begin{tabular}{ |p{2.5cm}||p{3.2cm}|p{2.5cm}|p{1.5cm}||p{3.5cm}|  }
 \hline
 \multicolumn{5}{|c|}{Posibles Riesgos} \\
 \hline
 Riesgo& Tipo & Probabilidad & Impacto & Descripción\\
 \hline
 Rotación del Personal& EXTERNO& ALTA & 3 &Personal abandona el proyecto\\
 Retraso de las Especidicaciones& ORGANIZACIÓN  & ALTA & 2 & Las especificaciones de las interfaces no estaran a tiempo\\
 Perdida de vidas humanas &EXTERNO & BAJA & 2 & Una persona del proyecto murio o un familiar de este\\
 Defectos en diseño & ORGANIZACIÓN &ALTA & 2 & El diseño no es el correcto\\
 Habilidades & EQUIPOS &MUY ALTA & 3 & El personal no esta del todo capacitado\\
 Enfermedad & EXTERNO & ALTA & 3 & Que el personal del proyecto enferme por covid u otra enfermedad\\
 Tiempo & ORGANIZACIÓN & ALTA & 2 & El tiempo requerido del proyecto este subestimado\\
 Economico &TECNOLOGIA & ALTA & 4 & Los integrantes del equipo no cuentan con suficiente dinero para comprar dispostitivos adecuados\\
 Falla de Dispositivo & TECNOLOGIA & MUY ALTA & 2 & La computadora de un integrante puede fallar\\
 Peridida de Informacion & INFORMACIÓN & ALTA & 2& No se guardo la informacion correctamente\\
 Organización & ORGANIZACIÓN & ALTA & 2 & El equipo no pueda organizarse ya que no hay comunicación y no tengan tiempo para terminar el proyecto\\ 
 \hline
\end{tabular}

\section{Conclusión}
En la gestion de riesgos podemos identificar los posibles problemas o riesgos que se pueden presentar y asi poder tener una buena organización para tomar medidas y prevenir los riesgos a los que se puede exponer nuestro proyecto asi este funcionara correctamente de acuerdo a las especificaciones que fue diseñado.\\


\section{Referencias}
\href{https://issuu.com/xaviernievez/docs/cuarto_docum_-_plan_de_gestion_de_riesgos}{Plan de Gestion}\\
\href{http://www.lsi.us.es/docencia/get.php?id=7133}{Gestion y Supervición}\\
\href{https://ocw.unican.es/pluginfile.php/1408/course/section/1803/tema7-gestionRiesgos.pdf}{Gestion de Riesgos}\\

\end{document}