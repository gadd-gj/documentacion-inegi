\title{Documento de visión y alcance }
\author{
        Gustavo U. Montes Hernandez \\
                Lic. Ingenieria de Software\\
        Universidad Veracruzana\\
        Campus Ixtaczoquitlán 
}
\date{\today}
\newpage

\documentclass[12pt]{article} 

\begin{document}
\newpage
\maketitle

\newpage \tableofcontents

\newpage \section{Visión de la solución}

Aqui veremos la vision del proyecto con sus respectivas caracteristicas de un mapa
en linea con su respectiva base de datos
 
\subsection{Declaración de la visión}\label{ant}

Un mapa que muestre cuantos comercios pequeños hay por estado y que tipo de comercio es. Con una 
pagina que solicite un registro para poder acceder.

\subsection{Características principales}\label{on}

Un mapa en linea que cuente con las siguientes caracteristicas:
\begin{itemize}

\item Iniciar con un registro.        
\item Mostrar mapa 
\item Leer base de datos de mas de 1,000,000.00 de registros

\end{itemize}

\subsection{Suposiciones y dependencias}\label{occe}

Supuestos:
\begin{enumerate}
        \item El hardware para visualizar sera proporcionado por el cliente.
        \item Se podra visualizar en cualquier explorador web moderno.
        \item El internet sera proporcionado por el cliente.
\end{enumerate}

Dependencias:
\begin{enumerate}
        \item La pagina se podra ver en cualquier sistema operativo mediante el navegador.
        \item Se podra visualizar en cualquier explorador web moderno.
        \item Esta diseñado para trabajar con pantallas touch.
\end{enumerate}

\section{Alcance y limitaciones}

Aqui veremos el alcance y limitaciones de este proyecto
 
\subsection{Alcance de la versión inicial}\label{ant}

Tener una pagina en linea, que mediante un registro pueda acceder a la vista inicial del mapa sin ninguna
funcionalidad

\subsection{Alcance de las versiones posteriores}\label{on}

Metas:
\begin{itemize}

        \item Usar la base de datos.       
        \item Hacer un conteo automatico de que tipo de comercio y en donde esta.
        \item Mostrar de manera didactica en mapa la informacion.
        \item Poder acceder mediante un inicio de sesion.
        \item Poder hacer un conteo selectivo por comercio.
        \item Distintos colores en el mapa dependiendo del comercio.
        
\end{itemize}


\subsection{Limitaciones y exclusiones}\label{occe}

Nuestras limitaciones son aquellos comercios que no fueron registrados, la imposibilidad de realizar una busqueda de aquellos locales 

\end{document}