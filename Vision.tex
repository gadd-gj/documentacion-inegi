\title{Documento de visión y alcance }
\author{
        Gustavo U. Montes Hernandez \\
                Lic. Ingenieria de Software\\
        Universidad Veracruzana\\
        Campus Ixtaczoquitlán 
}
\date{\today}
\newpage

\documentclass[12pt]{article} 

\begin{document}
\newpage
\maketitle

\newpage \tableofcontents

\newpage \section{Visión de la solución}

Aqui veremos la vision del proyecto con sus respectivas caracteristicas
 
\subsection{Declaración de la visión}\label{ant}

Un mapa que muestre cuantos comercios pequeños hay por estado y que tipo de comercio es.

\subsection{Características principales}\label{on}

Leer una base de datos de mas de 1,000,000 de comercios pequeños y separe automaticamente el tipo de comercio pequeño, lo cuente y lo muestre por medio de un mapa interactivo

\subsection{Suposiciones y dependencias}\label{occe}

Esta hecho para que se pueda ver desde cualquier navegador ya sea dispositivo movil y pc

\section{Alcance y limitaciones}

Aqui veremos el alcance y limitaciones de este proyecto
 
\subsection{Alcance de la versión inicial}\label{ant}

Tener un mapa que se pueda visualizar de manera online que cuente una base de datos de mas de 1,000,000 datos que separe, ceunte y asigne el numero de comercios por estado

\subsection{Alcance de las versiones posteriores}\label{on}

Un mapa que por medio de una base de datos 

\subsection{Limitaciones y exclusiones}\label{occe}

Nuestras limitaciones son aquellos comercios que no fueron registrados, la imposibilidad de realizar una busqueda de aquellos locales 




\end{document}