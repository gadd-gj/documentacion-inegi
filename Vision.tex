\title{Documento de visión y alcance }
\author{
        Arely Colorado Rodriguez \\
        Gaddiel Gomez Jimenez \\
        Citlalli Itzel Arceo García \\
        Gustavo U. Montes Hernandez \\
        David Freeman Cruz \\
        Gustavo Romero Oltehua \\
        \\
                Lic. Ingenieria de Software\\
        Universidad Veracruzana\\
        Campus Ixtaczoquitlán 
}
\date{\today}
\newpage

\documentclass[12pt]{article} 

\begin{document}
\newpage
\maketitle

\newpage \tableofcontents

\newpage \section{Visión de la solución}

El mapa INEGI de comercios pequeños ayuda a aumentar la eficiencia de como se analizan los datos de los comercios pequeño.\\

La division por categorias y la cantidad por estado con marcadores especiales de distintos colores.\\

Ayudara para una mejor administracion de los negocios pequeños y evitar la sobresaturacion de un mismo tipo de negocion en un mismo lugar, tambien ayudara a hacer estudios de mercado sobre que tipo de negocio y en donde abunda mas.
 
\subsection{Declaración de la visión}\label{ant}

Un mapa que muestre cuantos comercios pequeños hay por estado y que tipo de comercio es. Con una 
pagina que solicite un registro para poder acceder.

Mediante marcadores de colores para cada tipo de negocio.

\subsection{Características principales}\label{on}

Un mapa en linea que cuente con las siguientes caracteristicas:
\begin{itemize}

\item Iniciar con un registro.        
\item Mostrar mapa 
\item Leer base de datos de mas de 1,000,000.00 de registros

\end{itemize}

\subsection{Suposiciones y dependencias}\label{occe}

Supuestos:
\begin{enumerate}
        \item El hardware para visualizar sera proporcionado por el cliente.
        \item Se podra visualizar en cualquier explorador web moderno.
        \item El internet sera proporcionado por el cliente.
\end{enumerate}

Dependencias:
\begin{enumerate}
        \item La pagina se podra ver en cualquier sistema operativo mediante el navegador.
        \item Se podra visualizar en cualquier explorador web moderno.
        \item Esta diseñado para trabajar con pantallas touch.
\end{enumerate}

\section{Alcance y limitaciones}

Aqui veremos el alcance y limitaciones de este proyecto que planea mostrar la cantidad de comercios pequeños por estado, categorizarlo, y mediante marcadores de colores detectar el tipo de comercio y la cantidad.\\

Tambien que pueda mostrar graficas de cuantos negocios de un mismo tipo hay por estado para un correcto analisis de los datos.
 
\subsection{Alcance de la versión inicial}\label{ant}

Tener una pagina en linea, que mediante un registro pueda acceder a la vista inicial del mapa sin ninguna funcionalidad.

\subsection{Alcance de las versiones posteriores}\label{on}

Se espera a mediano alcanzar esta metas:
\begin{itemize}

        \item Usar la base de datos.       
        \item Hacer un conteo automatico de que tipo de comercio y en donde esta.
        \item Mostrar de manera didactica en mapa la informacion.
        \item Poder acceder mediante un inicio de sesion.
        \item Poder hacer un conteo selectivo por comercio.
        \item Distintos colores en el mapa dependiendo del comercio.
        \item Marcadores de colores en el mapa por negocio
        \item Poder hacer graficas con la informacion para un analisis correcto
        
\end{itemize}


\subsection{Limitaciones y exclusiones}\label{occe}

Nos limita la velocidad en la que se registran y publican las bases de datos actualizadas de comercios pequeño. Tambien para poder hacer un mapa mas completo nos detiene la informacion disponible a nuestra disposicion.\\

Tambien se excluyen negocios pequeños de temas extraños y aun no categorizados ya que no se registran de manera adecuada.

\end{document}